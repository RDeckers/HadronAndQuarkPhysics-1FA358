\input{../documents-common/preamble.tex}
\begin{document}
  \maketitle[Assignment 1]{Quark and Hadron Physics}
  \section{Relativity and Electrodynamics}
    \subsection{Four-derivative}
      Using Feynmann subscript notation:
      \begin{align}
        \partial^\mu\left(x\cdot b\right) &= \partial^\mu\left(x^\nu b_\nu\right),\nn
        &= \partial^\mu_x \left(x\cdot b) + \partial^\mu_b \left(x\cdot b),\nn
        &= (1, -\vec{1}) \cdot (b^0, \vec b) + (\pdrv[t] {b^0}, -\nabla \vec b)\cdot (t, \vec x),\nn
        &= (b^0, -\vec b) + 0 \cdot x, \nn
        &= b^\mu.
      \end{align}
      Note that we have had to assume that $b$ is independent of $t$ and $x$ to obtain this result. (only $b$ independent of $x$ was given).
    \subsection{Relativistic formulation of the Maxwell equations}
    If we expand
    \begin{align}
      \partial_\mu F^{\mu\nu} = j^\nu\nn
    \end{align}
    into it's matrix form we obtain
    \begin{align}
        \nabla \cdot E = \rho
    \end{align}
   In the first component ($\nu = 0$), which is Guass' law. We now turn our attention to the three
   other components ($\nu \in \left{1,2,3\right}$), expanded they read
   \begin{align}
     -\pdrv[t] E^1 + \pdrv[x_2]B^3 - \pdrv[x_3]B^2 = j^1\nn
     -\pdrv[t] E^2 + \pdrv[x_3]B^1 - \pdrv[x_1]B^3 = j^2\nn
     -\pdrv[t] E^3 + \pdrv[x_1]B^2 - \pdrv[x_2]B^1 = j^3
   \end{align}
   which can be written simultanously in vector form as
   \begin{align}
     \pdrv[t]{} \vec E =  \nabla \cross \vec B - \vec j,
   \end{align}
   which is Ampere's law.
   \begin{align}
     F_{\alpha\beta} &= g_{\alpha\mu}g_{\beta\nu}F^{\mu\nu}\nn
     &=g_{\alpha 0}g_{\beta 0}F^{00} +g_{\alpha i}g_{\beta 0}F^{i0} + g_{\alpha 0}g_{\beta i}F^{0i}+g_{\alpha i}g_{\beta j}F^{ij}
   \end{align}
   where $i,j \in 1..3$.
\end{document}
