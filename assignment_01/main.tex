\input{../documents-common/preamble.tex}
\begin{document}
  \maketitle[Assignment 1]{Quark and Hadron Physics}
  \section{Relativity and Electrodynamics}
    \subsection{Four-derivative}
      Using Feynmann subscript notation:
      \begin{align}
        \partial^\mu\left(x\cdot b\right) &= \partial^\mu\left(x^\nu b_\nu\right),\nn
        &= \partial^\mu_x \left(x\cdot b) + \partial^\mu_b \left(x\cdot b),\nn
        &= (1, -\vec{1}) \cdot (b^0, \vec b) + (\pdrv[t] {b^0}, -\nabla \vec b)\cdot (t, \vec x),\nn
        &= (b^0, -\vec b) + 0 \cdot x, \nn
        &= b^\mu.
      \end{align}
      Note that we have had to assume that $b$ is independent of $t$ and $x$ to obtain this result. (only $b$ independent of $x$ was given).
    \subsection{Relativistic formulation of the Maxwell equations}
    If we expand
    \begin{align}
      \partial_\mu F^{\mu\nu} = j^\nu\nn
    \end{align}
    into it's matrix form we obtain
    \begin{align}
        \nabla \cdot E = \rho
    \end{align}
   In the first component ($\nu = 0$), which is Guass' law. We now turn our attention to the three
   other components ($\nu \in \left{1,2,3\right}$), expanded they read
   \begin{align}
     -\pdrv[t] E^1 + \pdrv[x_2]B^3 - \pdrv[x_3]B^2 = j^1\nn
     -\pdrv[t] E^2 + \pdrv[x_3]B^1 - \pdrv[x_1]B^3 = j^2\nn
     -\pdrv[t] E^3 + \pdrv[x_1]B^2 - \pdrv[x_2]B^1 = j^3
   \end{align}
   which can be written simultanously in vector form as
   \begin{align}
     \pdrv[t]{} \vec E =  \nabla \cross \vec B - \vec j,
   \end{align}
   which is Ampere's law.
   Now, we note that:
   \begin{align}
     \epsilon^{\mu\nu\alpha\beta} \partial_\nu F_{\alpha\beta} &= 0 \nn
     &= \epsilon^{\mu\nu\alpha\beta}\partial_\nu \left(g^{\mu\alpha}g^{\nu\beta}F^{\mu\nu}\right)
   \end{align}
   Contracting the term in parenthesis gives us an explicit expression for $F_{\alpha\beta}$, namely,
   \begin{align}
     F_{\alpha\beta} =
     \begin{pmatrix}
       0 & -E^1 & - E^2 & -E^3\\
       E^1 & 0 & B^3 & -B^2\\
       E^2 & -B^3 & 0 & B^1\\
       E^3 & B^2 & -B^1 & 0
     \end{pmatrix}
   \end{align}
   If we now consider the case for $\mu = 0$ and contract the remaining variables we find
   \begin{align}
     \epsilon^{0\nu\alpha\beta}\partial_\nu F_{\alpha\beta} &= \partial_1\left(F_{23} - F_{32}\right) + \partial_2\left(F_{31}-F_{13}\right)+\partial_3\left(F_{12}-F_{21}\right)\nn
     &= \Aboxed{2\nabla\cdot\vec B = 0}
   \end{align}
   Taking $\mu = 1$ and contracting once more gives us
   \begin{align}
    \epsilon^{1\nu\alpha\beta}\partial_\nu F_{\alpha\beta} &= -\partial_0F_{23}+\partial_0F_{32} +\partial_2F_{03} - \partial_2F_{30} -\partial_3F_{02}+\partial_3F_{20}\nn
    &= \partial_0 B^1 +\partial_2E^3 -\partial_3E^2\nn
    &= \left(\pdrv[t]{\vec B} + \nabla\cross\vec E\right)^1
   \end{align}
   doing the same for $\mu = 2$ and $\mu = 3$ we find the other spatial coordinates of the above expression, leading us to conclude
   \begin{align}
      \left(\pdrv[t]{\vec B} + \nabla\cross {\vec E}\right) = 0
   \end{align}
   Finally, from $\partial_\mu j^\mu = 0$ it immidiatly follows from contraction that
   \begin{align}
     \pdrv[t]\rho = \nabla\cdot\vec j = 0.
   \end{align}
   \subsection{Electromagnetic potential}
   Given
   \begin{align}
     F^{\mu\nu}= \partial^\mu A^\nu - \partial^\nu A^\mu
   \end{align}
   I should be noted that both sides of this equation are antisymmetric so $\mu\nu = 01 = 10, \mu\nu = 23 = 32$ etc.
   If we take $\mu = 0$ and contract over $\nu$ we obtain:
   \begin{align}
    \vec E = -\pdrv[t]{} \vec A-\nabla\phi.
   \end{align}
   And the same for $\mu\nu = \mu 0$.
   \par Now taking all permutations for which $\mu < \nu$ ($\mu = \nu$ is the trivial case) we find:
   \begin{align}
     F^{12} = -B^3 &=\partial^1A^2-\partial^2A^1\nn
     F^{13} = B^2 &= \partial^1A^3-\partial^3A^1\nn
     F^{23} = B^1 &= \partial^2A^3-\partial^3A^2
   \end{align}
   which can be collected to form the vector expression
   \begin{align}
     \vec B = \nabla \cross \vec A
   \end{align}
   \subsection{Principle of minimal substitution}
   If
   \begin{align}
     H = \frac{1}{2m}\left(\vec p^2+q^2\vec A^2-2q\vec p \cdot \vec A\right)+q\phi
   \end{align}
    then
    \begin{align}
      \dot x^i = v^i = \pdrv[p^i]H \to \frac{1}{m}\left(\vec p -q\vec A\right)
    \end{align}
    which can be rewriten as
   \begin{align}
     \vec p = m \vec v + q \vec A
   \end{align}
    When we derive this expression w.r.t. time it leads to
  \begin{align}
     m\vec a = \vec{{\dot p}}-q\drv[t]{\vec A}
  \end{align}
    Using the second Hamiltonian equation we find that
    \begin{align}
      {\dot p}_i = -\pdrv[\vec x]H &= \frac{q}{m}\left(\vec p -\vec A\right)\cdot \pdrv[x_i]{\vec A} - q\pdrv[x_i]{\vec \phi}\nn
      &= q\vec v\cdot \pdrv[x_i]{\vec A} - q\pdrv[x_i]{\vec \phi}
    \end{align}
    Combining these last two equations leads to
    \begin{align}
      m a_i &= q\vec v\cdot \pdrv[x_i]{\vec A} - q\pdrv[x_i]{\vec \phi}-q\left(\drv[t]{\vec A}\right)_i\nn
      &=q\vec v\cdot \pdrv[x_i]{\vec A} - q\pdrv[x_i]{\vec \phi}-q\left(\pdrv[t]{A_i} + \vec v\cdot\nabla A_i\right)
    \end{align}
    We have previously shown this can be rewritten as:
    \begin{align}
      ma_i &= qE_i -q \left(\left(\vec v\cdot\nabla\right)A_i-\vec v\cdot\pdrv[x_i]{\vec A}\right)\nn
      &= qE_i -q v_j\left(\pdrv[x_j]{A_i} - \pdrv[x_i]{A_j}\right)\nn
      &= qE_i -q v_j\left(\partial^j A^i - \partial^i A^j\right)\nn
      ma_i &= q\left(E_i - v \cross \vec B\right)_i
    \end{align}
    \section{Three-body phase space}
    \subsection{}
    \begin{align}
      \frac{1}{2}\left(m_{12}^2-m_1^2-m_2^2\right) &= \frac{1}{2}\left(\left(p_1+p_2\right)^2-m_1^2-m_2^2\right)\nn
      &= \frac{1}{2}\left(p_1^2+p_2^2+2p_1\cdot p_2-m_1^2-m_2^2\right) \nn
      &= p_1\cdot p_2\nn
      \to & \frac{1}{2}\left(m_{23}^2-m_2^2-m_3^2\right) = p_2\cdot p_3
    \end{align}
    \begin{align}
      p_1\cdot p_3 7 &= \frac{1}{2}\left(M^2-m_{12}^2-m_{23}^2+m_2^2\right)\nn
      &= \frac{1}{2}\left(\left(p_1+p_2+p_3\right)^2-\left(p_1+p_2\right)^2-\left(p_2+p_3\right)^2+m_2^2\right)\nn
      &= \frac{1}{2}\left( m_1^2+m_2^2+m_3^2+2 p_1\cdot p_2 +2 p_1\cdot p_3 + 2p_2\cdot p_3 - m_2^2 - m_3^2 -2 p_2\cdot p_3 -m_1^2 -m_2^2 - 2 p_1\cdot p_2 + m_2^2\right)\nn
      &= \frac{1}{2}\left(2 p_1\cdot p_3\right)
    \end{align}
    \begin{align}
      \left(p_1+p_3\right)^2 &= m_1^2+m_3^2+2p_1\cdot p_3\nn
      &= m_1^2+m_3^2 + 2*\frac{1}{2}\left(M^2-m_{12}^2-m_{23}^2+m_2^2\right)\nn
      &= M^2-m_{12}^2-m_{23}^2+m_1^2+m_2^2+m_3^2
    \end{align}
\end{document}
